\documentclass{article}
\usepackage{graphicx}
\usepackage{alltt}
\usepackage{amsmath}
\usepackage{amsfonts}
\usepackage{bigstrut}
\usepackage{enumerate}
\usepackage{fancyhdr}
\usepackage[top=.75in, bottom=.95in, left=.75in, right=.75in]{geometry}
\usepackage{float}
\usepackage{lastpage}
\usepackage{tikz}
\usepackage[latin1]{inputenc}
\usepackage{color}
\usepackage{array}
\usepackage{longtable}
\usepackage{calc}
\usepackage{multirow}
\usepackage{hhline}
\usepackage{ifthen}
\usepackage{listings}
\usepackage{circuitikz}
\usepackage{caption}
\definecolor{mygreen}{rgb}{0,0.6,0}
\definecolor{mygray}{rgb}{0.5,0.5,0.5}
\definecolor{mymauve}{rgb}{0.58,0,0.82}
\lstset{ %
  backgroundcolor=\color{white},   % choose the background color; you must add \usepackage{color} or \usepackage{xcolor}; should come as last argument
  basicstyle=\normalsize,        % the size of the fonts that are used for the code
  breakatwhitespace=false,         % sets if automatic breaks should only happen at whitespace
  breaklines=true,                 % sets automatic line breaking
  captionpos=b,                    % sets the caption-position to bottom
  commentstyle=\color{mygreen},    % comment style
  deletekeywords={...},            % if you want to delete keywords from the given language
  escapeinside={\%*}{*)},          % if you want to add LaTeX within your code
  extendedchars=true,              % lets you use non-ASCII characters; for 8-bits encodings only, does not work with UTF-8
  frame=single,	                   % adds a frame around the code
  keepspaces=true,                 % keeps spaces in text, useful for keeping indentation of code (possibly needs columns=flexible)
  keywordstyle=\color{blue},       % keyword style
  language=python,                  % the language of the code
  morekeywords={*,...},            % if you want to add more keywords to the set
  numbers=left,                    % where to put the line-numbers; possible values are (none, left, right)
  numbersep=5pt,                   % how far the line-numbers are from the code
  numberstyle=\tiny\color{mygray}, % the style that is used for the line-numbers
  rulecolor=\color{black},         % if not set, the frame-color may be changed on line-breaks within not-black text (e.g. comments (green here))
  showspaces=false,                % show spaces everywhere adding particular underscores; it overrides 'showstringspaces'
  showstringspaces=false,          % underline spaces within strings only
  showtabs=false,                  % show tabs within strings adding particular underscores
  stepnumber=2,                    % the step between two line-numbers. If it's 1, each line will be numbered
  stringstyle=\color{mymauve},     % string literal style
  tabsize=2,	                   % sets default tabsize to 2 spaces
  title=\lstname                   % show the filename of files included with \lstinputlisting; also try caption instead of title
}
\floatstyle{boxed}
\floatstyle{plain}
\restylefloat{figure}
\pagestyle{fancy}
\fancyhead{}
\fancyfoot{}
\setlength{\headheight}{59.0pt}
\def\inputGnumericTable{}
\fancyhead[CO]{\textbf{Air Force Institute of Technology\\Department of Electrical and Computer Engineering\\
 Computer Communication Networks (CSCE-654) Project \#2\newline \newline Name: Micah Hayden}}
\lhead{\today}
\rhead{Page \thepage{} of \pageref{LastPage} }
\newlength\tindent
\setlength{\tindent}{\parindent}
\setlength{\parindent}{0pt}
\renewcommand{\indent}{\hspace*{\tindent}}
\begin{document}

%\begin{abstract}
%This is my abstract.
%\end{abstract}

\section{Introduction:}
\label{sec:Intro}


\section{Expected Results:}
\label{sec:ExpectedResults}


\section{Simulation Setup:}
\label{sec:SimSetup}


\subsection{Routing:}
\label{subsec:Routing}


\subsection{Network Configuration:}  
\label{subsec:Config}


\section{Results \& Analysis:}
\label{sec:Results}


\subsection{Routing Accuracy \& Traffic Load:}
\label{subsec:Routing}


\subsection{System Delay:}
\label{subsec:SysDelay}




\section{Conclusions:}
\label{sec:Conclusions}

This project demonstrated the long term effects of a queuing system for different values of $\rho$.  
When $\lambda$ is much less than $\mu$, the system is able to keep up, and produces minimal delays.  
As $\lambda$ approaches $\mu$, the queue will increase in length; however, as long as $\lambda < \mu$, the system will still be stable.
Once $\lambda = \mu$, the system becomes unstable, and will have a system delay and queue length that will grow indefinitely.

Another interesting outcome of this experiment was seeing the effects of the propagation delay.  
One would expect the two queues to have identical times in the long run; however, they were slightly different for all simulations.
Because of these variations, using the average queue times and average lifetime to calculate the propagation delay produced a slightly different value than the constant specified in the simulation, as shown in Table \#\ref{tab:calcPropDelay}.
This shows the inherent complexity of a system of queues, even when the queues have the same parameters for expected arrival rates and service rates.


\newpage
\section*{Appendix A:  Figures}
\label{sec:Figures}



\newpage
\clearpage
\section*{Appendix B:  Simulation Files}
\label{sec:SimFiles}

\begin{figure}[h!]
\begin{lstlisting}

\end{lstlisting}
\vspace{-1cm}
\caption*{Simulation Initialization File - omnetpp.ini}
\end{figure}


\newpage
\begin{figure}[h!]
\begin{lstlisting}

\end{lstlisting}
\vspace{-1cm}
\caption*{NED FILE}
\end{figure}

\end{document}
